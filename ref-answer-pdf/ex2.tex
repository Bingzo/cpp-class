%\documentclass[aps,twocolumn,prb,superscriptaddress]{revtex4}     % for final submission
%\documentclass[aps,showpacs,prb]{revtex4}
\documentclass[14pt,aps,prb]{revtex4}
%\documentclass[12pt]{article}

\usepackage{graphicx}
\usepackage{setspace}
\usepackage{bm}
\usepackage{amsmath,amssymb,amsthm}
\usepackage{multirow}
\usepackage{hyperref}
\usepackage{float}
\usepackage{listings}
\usepackage{xcolor}

 \usepackage{ctex}

\begin{document}
\begin{spacing}{1.5}

\title{C++ 第二次实验参考答案}
\author{周兵}

\date{\today}% It is always \today, today,
             %  but any date may be explicitly specified
\maketitle

\lstset{
    language=C++,
    basicstyle=\ttfamily\small,
    breaklines=true,
    prebreak=\raisebox{0ex}[0ex][0ex]{\ensuremath{\hookleftarrow}},
    frame=lines,
    showtabs=false,
    showspaces=false,
    showstringspaces=false,
    keywordstyle=\color{red}\bfseries,
    stringstyle=\color{green!50!black},
    commentstyle=\color{gray}\itshape,
%    numbers=left,
    captionpos=t,
    escapeinside={\%*}{*)}
}

\textcolor{red}{注:练习时,代码一定要对得很整齐。这个习惯一定要养成。}

\textcolor{blue}{人民币那题很不用数组的话只能分前四位后四位穷举,比较繁琐;此参考答案中使用了二维数组。}

1、输入一个数,判断它的奇偶性后输出结果。
\begin{lstlisting}[language=C++]
//if else
#include <iostream>
#include <cmath>
using namespace std;

int main(){
    int n;
    cout << "Pleased input a integer: ";
    cin >> n;
    if( 0 == n % 2 ){
        cout << "Even number." << endl;
    }
    else{
        cout << "Odd number." << endl;
    }
	
    return 0;
}

// ? :
#include <iostream>
#include <cmath>
using namespace std;

int main(){
    int n;
    cout << "Pleased input a integer: ";
    cin >> n;
    n % 2 == 0 ? cout << "Even number." : cout << "Odd number.";
    cout << endl;
	
    return 0;
}
\end{lstlisting}

2、编程求一元二次方程$ax^2+bx+c=0$的根。包括以下判断和结果,若输入$a=0$,给出提示;$\Delta=b^2-4ac $,若$\Delta>0$,输出两个不等的实根;
\begin{lstlisting}[language=C++]
#include <iostream>
#include <cmath>
using namespace std;

int main(){
    int a, b, c;
    cout << "Please input three integer parameters for the equation(a, b, c): ";
    cin >> a >> b >> c;
    int delta = b*b - 4*a*c;
    double x1, x2;

    if( 0 == a ){
        cout << "Illegal input, a can't be zero." << endl;
        return -1;
    }
    else{
        if( 0 == delta ){
            x1 = x2 = -0.5*b/a;
            cout << " Equation has the same 2 roots: " << endl;
            cout << '\t' << x1 << '\t' << x2 << endl;
        }
        else if( delta > 0 ){
            x1 = 0.25/a * ( -b + sqrt( (double)delta ) );
            x2 = 0.25/a * ( -b - sqrt( (double)delta ) );
            cout << " Equation has 2 different roots: " << endl;
            cout << '\t' << x1 << '\t' << x2 << endl;
        }
        else{
            double real, imag;
            real = 0.25/a * ( -b );
            imag = 0.25/a * sqrt( -(double)delta );
            cout << " Equation has 2 imaginary roots: " << endl;
            cout << '\t' << real << '+' << imag << 'i';
            cout << '\t' << real << '-' << imag << 'i' << endl;
        }
    }

    return 0;
    
output:
Please input three integer parameters for the equation(a, b, c): 1 2 1
 Equation has the same 2 roots:
        -1      -1
Please input three integer parameters for the equation(a, b, c): 1 2 3
 Equation has 2 imaginary roots:
        -0.5+0.707107i  -0.5-0.707107i
}
\end{lstlisting}

3、编写程序:输入一门课程的成绩,若高于90分,输出“A  grade ”;若高于80分而低于90分,输出“B grade ”;若高于70分而低于80分,输出“C  grade ”;若高于60分而低于70分,输出“D  grade ”;否则输出“Not passed ”。
\begin{lstlisting}[language=C++]
//if else
#include <iostream>
#include <cmath>
using namespace std;

int main(){
    int mark;
    cout << "Please input the mark (integer, 0~100): ";
    cin >> mark;

    if( mark >= 90 ){
        cout << "A grade" << endl;
    }
    else if( mark >= 80 ){
        cout << "B grade" << endl;
    }
    else if( mark >=70 ){
        cout << "C grade" << endl;
    }
    else if( mark >= 60 ){
        cout << "D grade" << endl;
    }
    else{
        cout << "Not passed" << endl;
    }

    return 0;
}

//switch case
#include <iostream>
#include <cmath>
using namespace std;

int main(){
    int mark;
    cout << "Please input the mark (integer, 0~100): ";
    cin >> mark;

    mark /= 10;

    switch( mark ){
    case 9:
        cout << "A grade" << endl;
        break;
    case 8:
        cout << "B grade" << endl;
        break;
    case 7:
        cout << "C grade" << endl;
        break;
    case 6:
        cout << "D grade" << endl;
        break;
    default:
        cout << "Not passed" << endl;
    }

    return 0;
}

// array
#include <iostream>
#include <cmath>
using namespace std;

int main(){
    int mark;
    cout << "Please input the mark (integer, 0~100): ";
    cin >> mark;

    mark /= 10;

    char grade[4] = { 'A', 'B', 'C', 'D' };
    if( mark > 5 )
        cout << grade[9-mark] << " grade" << endl;
    else
        cout << "Not passed" << endl;

    return 0;
}
\end{lstlisting}

4、编写程序: 输入一个数,判断其是否是3或7的倍数,可分为4种情况输出。
(1) 是3的倍数,但不是7的倍数。 
(2) 不是3的倍数,是7的倍数。
(3) 是3的倍数,也是7的倍数。 
(4) 既不是3的被数,也不是7的倍数。
\begin{lstlisting}[language=C++]
// if else
#include <iostream>
#include <cmath>
using namespace std;

int main(){
    int num;
    cout << "Please input a number (integer): ";
    cin >> num;

    if( 0 == num%3 && 0 != num%7 ){
        cout << "是3的倍数,但不是7的倍数" << endl;
    }
    else if( 0 != num%3 && 0 == num%7 ){
        cout << "不是3的倍数,是7的倍数" << endl;
    }
    else if( 0 == num%3 && 0 == num%7 ){
        cout << "是3的倍数,也是7的倍数" << endl;
    }
    else if( 0 != num%3 && 0 != num%7 ){
        cout << "既不是3的被数,也不是7的倍数" << endl;
    }
    return 0;
}

//
#include <iostream>
#include <cmath>
using namespace std;

int main(){
    int num;
    cout << "Please input a number (integer): ";
    cin >> num;

    if( 0 == num%3 ){
        if(  0 == num%7 )
            cout << "是3的倍数,是7的倍数" << endl;
        else
            cout << "是3的倍数,但不是7的倍数" << endl;
    }
    else{
        if(  0 == num%7 )
            cout << "不是3的倍数,是7的倍数" << endl;
        else
            cout << "既不是3的倍数,也不是7的倍数" << endl;
    }
    return 0;
}
\end{lstlisting}

1、输入三个数字a,b,c,
运用if-else语句分别判断a,b,c的大小。
\begin{lstlisting}[language=C++]
// if else
#include <iostream>
#include <cmath>
using namespace std;

int main(){
    int a, b, c;
    cout << "Please input three integer numbers (a, b, c): ";
    cin >> a >> b >> c;

    int max;
    int middle;
    int min;

    if( a > b && a > c ){
        max = a;
    }
    else if ( b > c ){
        max = b;
    }
    else{
        max = c;
    }

    if( a < b && a < c ){
        min = a;
    }
    else if ( b < c ){
        min = b;
    }
    else{
        min = c;
    }

    if( a < max && a > min ){
        middle = a;
    }
    else if( b < max && b > min ){
        middle = b;
    }
    else{
        middle = c;
    }
    cout << '\t' << max << " > " << middle << " > " << min << endl;
}

output:
Please input three integer numbers (a, b, c): 1 3 2
        3 > 2 > 1
\end{lstlisting}

2、运用switch语句分三种情况,运用while语句输入60次选择,最后加和,输出结果票数。
\begin{lstlisting}[language=C++]
// switch case
#include <iostream>
#include <cstdlib>
using namespace std;

int main(){
    int a, b, c;
    a = b = c = 0;

    cout << "Polling..." << endl;
    for( int i = 0; i < 30; i ++ ){
        for( int j = 0; j < 2; j ++ ){
            int poll = rand()%3 + 1;
            switch( poll ){
            case 1:
                ++ a;
                break;
            case 2:
                ++ b;
                break;
            case 3:
                ++ c;
                break;
            default:
                cout << "Illegal input. Negnected." << endl;
            }
        }
    }
    cout << " The votes are: " << endl;
    cout << '\t' << a << '\t' << b << '\t' << c << endl;
    
    return 0;
}

output:
Polling...
 The votes are:
        20      18      22
\end{lstlisting}

3、输入一个正整数n,运用for语句,输出阶乘结果。
\begin{lstlisting}[language=C++]
//long long
#include <iostream>
using namespace std;

int main(){
    int n;
    cout << "Input n (integer, <= 20): ";
    cin >> n;

    long long fp = 1;
    int i = n;
    while( i > 0 ){
        fp *= i;
        -- i;
    }

    cout << n << "! = " << fp << endl;
    
    return 0;
}

output:
Input n (integer, <= 20): 20
20! = 2432902008176640000
\end{lstlisting}

4、输入年份,月份,日期,运用if-else语句判断是否为闰年,闰年二月份为28天,运用for语句算出天数是本年第几天,输出结果。
\begin{lstlisting}[language=C++]
#include <iostream>
using namespace std;

int main(){
    int year, month, day;
    cout << "Input the date(year month day): ";
    cin >> year >> month >> day;

    int isLeap = 0;
    if( ( 0 == year % 4 && 0 != year %100 ) || 0 == year %400 ){
        isLeap = 1;
        cout << "Leap year." << endl;
    }

    int dayspermonth[12] = { 31, 0, 31, 30, 31, 30, 31, 31, 30, 31, 30, 31 };
    isLeap ? dayspermonth[1]= 29 : dayspermonth[1]= 28;

    int thisday = 0;
    for( int i = 0; i < month - 1; i ++ ){
        thisday += dayspermonth[i];
    }
    thisday += day;

    cout << "This is Day " << thisday << '.' << endl;
    
    return 0;
}
\end{lstlisting}

5、输入一个整数(位数不超过9位)代表一个人民币值(单位为元),请转换成财务要求的大写格式。如23108元,转换后变成“贰万叁仟壹佰零捌元”。200101元,转换后变成“贰拾万零壹佰零壹元”。
\begin{lstlisting}[language=C++]
#include <iostream>
#include <cmath>
using namespace std;

int main()
{
	char *one2ten = "零壹贰叁肆伍陆柒捌玖";
    char *units = "萬仟佰拾元";
    
    char one2ten_index[10][3];    // 零壹贰叁肆伍陆柒捌玖
    for( int i = 0; i < 10; i ++ ){
        for( int j = 0; j < 2; j ++ ){
            one2ten_index[i][j] = one2ten[i*2+j];
        }
        one2ten_index[i][2] = '\0';
        //cout << one2ten_index[i] << endl;
    }

    char units_index_0[5][3];    // 萬仟佰拾
    for( int i = 0; i < 5; i ++ ){
        for( int j = 0; j < 2; j ++ ){
            units_index_0[i][j] = units[i*2+j];
        }
        units_index_0[i][2] = '\0';
        //cout << units_index_0[i] << endl;
    }

    int money;
    cout << "输入一个小于9位的整数: ";
    cin >> money;
    if( money <=0 || money >= 1e8 ){
        cout << "输入错误" << endl;
        return -1;
    }

    int N = 8;
    while( 0 == money / (int)pow( 10, N) ){
        -- N;
    }
    ///++ N; // 获得输入数的位数

    int tail = money;
    int inc = N;
    int head;
    if( N > 3 ){
        while( inc > 4 ){
            head = tail / (int)pow( 10, inc );
            cout << one2ten_index[head];
            cout << units_index_0[8 - inc];
            tail %= (int)pow( 10, inc );
            -- inc;
        }
        head = tail / (int)pow( 10, inc );
        cout << one2ten_index[head];
        cout << units_index_0[0];
        tail %= (int)pow( 10, inc );
        -- inc;
    }
    while( inc >= 0 ){
        head = tail / (int)pow( 10, inc );
        cout << one2ten_index[head];
        cout << units_index_0[4-inc];
        tail %= (int)pow( 10, inc );
        -- inc;
    }
    
    cout << endl;

    return 0;
}
\end{lstlisting}

6、编写程序计算y=1-1/2+1/3-1/4+...+1/n
\begin{lstlisting}[language=C++]
#include <iostream>
using namespace std;

int main(){
    int n;
    cout << "Input n (integer, > 0): ";
    cin >> n;

    double sum = 1.0;
    int p = -1;
    for( int i = 1; i < n; i ++){
        sum += 1.0 * p / ( i + 1.0 );
        p = -p;
    }
    cout << "y = " << sum << endl;
    
    return 0;
}
\end{lstlisting}

%\begin{figure}
  % Requires \usepackage{graphicx}
%  \includegraphics[width=0.5\textwidth]{figures/exec_time.png}\\
%  \caption{Comparison of the execution time of LAMMPS using the rhodopsin protein benchmark (2048000
%atoms) on XCmaster, HPCx and Blue Gene CO and VN modes.}\label{fig:exec_time}
%\end{figure}

\end{spacing}
\end{document}
%
% ****** End of file apssamp.tex ******
