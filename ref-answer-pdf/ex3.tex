%\documentclass[aps,twocolumn,prb,superscriptaddress]{revtex4}     % for final submission
%\documentclass[aps,showpacs,prb]{revtex4}
\documentclass[14pt,aps,prb]{revtex4}
%\documentclass[12pt]{article}

\usepackage{graphicx}
\usepackage{setspace}
\usepackage{bm}
\usepackage{amsmath,amssymb,amsthm}
\usepackage{multirow}
\usepackage{hyperref}
\usepackage{float}
\usepackage{listings}
\usepackage{xcolor}

 \usepackage{ctex}

\begin{document}
\begin{spacing}{1.5}

\title{C++ 第三次实验参考答案(共14题)}
\author{周兵}

\date{\today}% It is always \today, today,
             %  but any date may be explicitly specified
\maketitle

\lstset{
    language=C++,
    basicstyle=\ttfamily\small,
    breaklines=true,
    prebreak=\raisebox{0ex}[0ex][0ex]{\ensuremath{\hookleftarrow}},
    frame=lines,
    showtabs=false,
    showspaces=false,
    showstringspaces=false,
    keywordstyle=\color{red}\bfseries,
    stringstyle=\color{green!50!black},
    commentstyle=\color{gray}\itshape,
%    numbers=left,
    captionpos=t,
    escapeinside={\%*}{*)}
}

%\textcolor{red}{注:练习时,代码一定要对得很整齐。这个习惯一定要养成。}

1、由0到4五个数字,组成5位数,每个数字用一次,但十位和百位不能为3(当然万位不能为0),输出所有可能的五位数。
\begin{lstlisting}[language=C++]
#include<iostream>
#include<cmath>
using namespace std;
int main()
{
    for( int i = 1; i < 5; i ++ ){ // 万位,不为0
        for( int j = 0; j < 5; j ++ ){ //千位
            if( j != i ){
                for( int k = 0; k < 5; k ++ ){ //百位
                    if( k != j && k != i && k != 3){ // 百位不为3
                        for( int h = 0; h < 5; h ++ ){ // 十位
                            if( h != k && h != j && h != i && h != 3){ // 十位不为3
                                for( int n = 0; n < 5; n ++ ){ // 个位
                                    if( n != h && n != k && n != j && n != i ){ 
                                        cout << 1e4*i + 1e3*j + 1e2*k + 1e1 *h + 1e0 * n << ' ';
                                    }
                                }
                            }
                        }
                    }
                }
            }
        }
    }
    cout << endl;

    return 0;
}
\end{lstlisting}

2、编程求和:s=a+aa+aaa+aaaa+ ……+aaaa…aaa(n个),其中a为1~9中的一个数字。
\begin{lstlisting}[language=C++]
#include<iostream>
#include<cmath>
using namespace std;
int main()
{

    int a, n;
    cout << "a = ";
    cin >> a;
    cout << "n = ";
    cin >> n;

    int s = 0;
    for( int i = 0; i < n; i ++ ){
        s += a;
        a = 10 * a + a;
    }
    
    cout << "s = " << s << endl;

    return 0;
}
\end{lstlisting}

3、编程求出所有的“水仙花数”。所谓“水仙花数”是指一个3位数,其中各位数字的立方和等于该数本身,例如153就是一个“水仙花数”,因为$153=1^3+5^3+3^3$。要求采用枚举法。
\begin{lstlisting}[language=C++]
#include<iostream>
#include<cmath>
using namespace std;
int main()
{
    for( int i = 100; i < 1000; i ++ ){
        int n = i; // n 获得 i 的值
        int s = 0; // 等号右边的和
        for( int j = 0; j < 3; j ++ ){
            int k = n % 10; // 获得 n 的最后一位
            n /= 10; // 舍去最后一位
            s += k*k*k; // 计算3次方
        }
        if( i == s ){
            cout << i << ' '; // 满足“水仙花”公式
        }
    }
    
    cout << endl;

    return 0;
}
\end{lstlisting}

4、编程计算求π的近似值,误差小于le-12
\begin{lstlisting}[language=C++]
#include<iostream>
#include<cmath>
#include<iomanip> // setprecision
using namespace std;
int main()
{
    double pi = 0.0; 
    int sign = 1; // 正负号
    int i = 0;
    double an; // 每项的值,不包括正负号
    do{
        an = 1.0 / ( 2*i+1 ); // 每项的值,不包括正负号
        pi += sign * an;
        sign = -sign;
        ++ i;
    } while( an > 1e-12 );
    pi *= 4;
    cout << setprecision(12) << "pi = " << pi << endl; // 输出精确到12位

    return 0;
}

output:
pi = 3.14159265079
\end{lstlisting}

5、连续输入n个整数(n由键盘输入),统计其中整数、负数和零的个数。
\begin{lstlisting}[language=C++]
#include<iostream>
#include<cmath>
using namespace std;
int main()
{
    int arr[100]; // 数组应足够大
    int n;

    cout << "n (n<100) = ";
    cin >> n; // 输入 n , n 应小于数组大小
    
    for( int i = 0; i < n; i ++ ){
        arr[i] = rand()%100 - 50; // 对数组使用随机数赋值,范围-50~49
    }

    int nNeg = 0; // 统计负数个数
    int nZero = 0; // 统计0的个数
    int nInt = 0; // 统计整数个数
    for( int i = 0; i < n; i ++ ){
        if( arr[i] < 0 ){
            ++ nNeg; // 统计负数个数
        }
        if( arr[i] == 0 ){
            ++ nZero; // 统计0的个数
        }
        ++ nInt; // 统计整数个数
    }

    cout << "整数" << nInt << "个;" <<endl;
    cout << "负数" << nNeg << "个;" <<endl;
    cout << "零" << nZero << "个。" <<endl;

    return 0;
}
\end{lstlisting}

6、请给10个随机整数,并按产生随机数的次序输出10个整数,再排序,并输出排序结果。
\begin{lstlisting}[language=C++]
#include<iostream>
#include<cmath>
using namespace std;
int main()
{
    int arr[10]; // 数组应足够大
    
    for( int i = 0; i < 10; i ++ ){
        arr[i] = rand()%100 - 50; // 对数组使用随机数赋值,范围-50~49
    }

    // 输出数组
    for( int i = 0; i < 10; i ++ ){
        cout << arr[i] << ' ';
    }
    cout << endl;

    // 冒泡排序
    for( int i = 0; i < 10-1; i ++ ){
        for( int j = 0; j < 10-1-i; j ++ ){
            if( arr[j] > arr[j+1] ){
                int x = arr[j];
                arr[j] = arr[j+1];
                arr[j+1] = x; 
            }
        }
    }

    // 输出数组
    for( int i = 0; i < 10; i ++ ){
        cout << arr[i] << ' ';
    }
    cout << endl;

    return 0;
}
\end{lstlisting}

7、请输出如下图形:\\
A\\
ABA\\
ABCBA\\
ABCDCBA\\
ABCDEDCBA
\begin{lstlisting}[language=C++]
#include<iostream>
#include<cmath>
using namespace std;
int main()
{
    char chr[5] = { 'A', 'B', 'C', 'D', 'E' };

    for( int i = 0; i < 5; i ++ ){
        for( int j = 0; j < i; j ++ ){
            cout << chr[j];
        }
        cout << chr[i];
        for( int j = i-1; j >= 0; j -- ){
            cout << chr[j];
        }
        cout << endl;
    }

    return 0;
}
\end{lstlisting}

8、求出水仙花数后不是在屏幕上显示而是存入文本文件。请在退出程序后,用记事本打开该文本文件,查看结果。
\begin{lstlisting}[language=C++]
#include<iostream>
#include<fstream>
#include<cmath>
using namespace std;
int main()
{
    ofstream ofs( "out.txt", ios_base::trunc );

    for( int i = 100; i < 1000; i ++ ){
        int n = i; // n 获得 i 的值
        int s = 0; // 等号右边的和
        for( int j = 0; j < 3; j ++ ){
            int k = n % 10; // 获得 n 的最后一位
            n /= 10; // 舍去最后一位
            s += k*k*k; // 计算3次方
        }
        if( i == s ){
            ofs << i << ' '; // 满足“水仙花”公式
        }
    }
    
    ofs << endl;
    ofs.close();

    return 0;
}
\end{lstlisting}

9、编程从上题生成的文本文件读取水仙花数,并显示在屏幕上。
\begin{lstlisting}[language=C++]
#include<iostream>
#include<fstream>
#include<cmath>
using namespace std;
int main()
{
    ifstream ifs( "out.txt", ios_base::in );

    char chr[1024];
    while( !ifs.eof() ){
        ifs.getline( chr, 1024 );
        cout << chr << endl;
    }

    ifs.close();

    return 0;
}
\end{lstlisting}

10、编写程序求500 以内的勾股弦数,即满足 $c^2=b^2+a^2$ (2为指数)的3个数,要求b>a。将所有符合要求的组合存入文本文件中。
\begin{lstlisting}[language=C++]
#include<iostream>
#include<fstream>
using namespace std;
int main()
{
    ofstream ofs( "out.txt", ios_base::trunc );
    for( int c = 1; c < 500; c ++ ){
        for( int b = 1; b < c; b ++ ){
            for( int a = 1; a < b; a ++ ){
                if( c*c == a*a + b*b ){
                    ofs << c << ' ' << b << ' ' << a << endl;
                }
            }
        }
    }
    ofs.close();

    return 0;
}
\end{lstlisting}

11、将十进制整数n转换为以十六进制为基数的数并输出。
\begin{lstlisting}[language=C++]
#include<iostream>
using namespace std;
int main()
{
    int n;
    cout << "n = ";
    cin >> n;

    char hex[3];
    int i = 0;
    do{
        hex[i] = n % 16;
        n /= 16;
        ++ i;
    }while( n != 0 );

    for( int j = 0; j < i; j ++ ){
        if( hex[j] < 10 ){
            hex[j] += '0';
            cout << hex[j];
        }
        else{
            hex[j] += 'A' - 10;
            cout << hex[j];
        }
    }
    cout << endl;

    return 0;
}
\end{lstlisting}

12、请输入任意二进制整数,转换成十进数并输出
\begin{lstlisting}[language=C++]
#include<iostream>
using namespace std;
int main()
{
    int n;
    cout << "n = ";
    cin >> n;

    int dec = 0;
    int i = 0;
    int power = 1;

    do{
        i = n % 10;
        n /= 10;
        dec += power * i;
        power *= 2;
    }while( n != 0 );

    cout << "The decimal number is: " << dec << endl;

    return 0;
}
\end{lstlisting}

13、请输入任意不多于20个数并输出,对这组数排序,并输出排序结果。
\begin{lstlisting}[language=C++]
//参考第6 题。
\end{lstlisting}

14、编程输出下列图形,中间一行英文字母由输入得到。
\begin{lstlisting}[language=C++]
/*
           A
        B  B  B
     C  C  C  C  C
  D  D  D  D  D  D  D
     C  C  C  C  C
        B  B  B
           A
*/
// 之前群里发过。
\end{lstlisting}
%\begin{figure}
  % Requires \usepackage{graphicx}
%  \includegraphics[width=0.5\textwidth]{figures/exec_time.png}\\
%  \caption{Comparison of the execution time of LAMMPS using the rhodopsin protein benchmark (2048000
%atoms) on XCmaster, HPCx and Blue Gene CO and VN modes.}\label{fig:exec_time}
%\end{figure}

\end{spacing}
\end{document}
%
% ****** End of file apssamp.tex ******
